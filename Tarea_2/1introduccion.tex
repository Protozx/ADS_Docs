%=========================================================
\chapter{Introducción}


\cdtInstrucciones{
	Presentar el documento, indicando su contenido, a quien va dirigido, quien lo realizó, por que razón, dónde y cuando. \\
}
Este documento contiene la Especificacion del ptoyecto ``{\em Nombre del proyecto}'' correspondiente al trabajo realizado en el semestre 2016-2017-2 para la materia de Análisis y diseño orientado a objetos en el grupo 2CV9 por el equipo {\em Nombre del equipo}.

%---------------------------------------------------------
\section{Presentación}


\cdtInstrucciones{
	Indique el propósito del documento y las distintas formas en que puede ser utilizado.\\
}
Este documento contiene la especificación de los requerimientos del usuario y del sistema del sistema a desarrollar. Tiene como objetivo establecer la naturaleza y funciones del sistema para su evaluación al final del semestre. Este documento debe ser aprobado por los principales responsables del proyecto.

Este documento es el C2-EP1 del proyecto ``{\em Nombre del proyecto}''.

%---------------------------------------------------------
\section{Organización del contenido}

En el capítulo \ref{cap:reqUsr} ...

En el capítulo \ref{cap:reqSist} ...

%---------------------------------------------------------
\section{Notación, símbolos y convenciones utilizadas}

Los requerimientos funcionales utilizan una clave RFX, donde:

\begin{description}
	\item[X] Es un número consecutivo: 1, 2, 3, ...
	\item[RF] Es la clave para todos los {\textbf{R}}equerimientos {\textbf{F}}uncionales.
\end{description}

Los requerimientos del usuario utilizan una clave RUX, donde:

\begin{description}
	\item[X] Es un número consecutivo: 1, 2, 3, ...
	\item[RU] Es la clave para todos los {\textbf{R}}equerimientos del {\textbf{U}}suario.
\end{description}

Además, para los requerimeitnos funcionales se usan las abreviaciones que se muestran en la tabla~\ref{tbl:leyendaRF}.
\begin{table}[hbtp!]
	\begin{center}
		\begin{tabular}{|r l|}
			\hline
			{\footnotesize Id}   & {\footnotesize\em Identificador del requerimiento.}            \\
			{\footnotesize Pri.} & {\footnotesize\em Prioridad}                                   \\
			{\footnotesize Ref.} & {\footnotesize\em Referencia a los Requerimientos de usuario.} \\
			{\footnotesize MA}   & {\footnotesize\em Prioridad Muy Alta.}                         \\
			{\footnotesize A}    & {\footnotesize\em Prioridad Alta.}                             \\
			{\footnotesize M}    & {\footnotesize\em Prioridad Media.}                            \\
			{\footnotesize B}    & {\footnotesize\em Prioridad Baja.}                             \\
			{\footnotesize MB}   & {\footnotesize\em Prioridad Muy Baja.}                         \\
			\hline
		\end{tabular}
		\caption{Leyenda para los requerimientos funcionales.}
		\label{tbl:leyendaRF}
	\end{center}
\end{table}